\PassOptionsToPackage{unicode=true}{hyperref} % options for packages loaded elsewhere
\PassOptionsToPackage{hyphens}{url}
%
\documentclass[]{article}
\usepackage{lmodern}
\usepackage{amssymb,amsmath}
\usepackage{ifxetex,ifluatex}
\usepackage{fixltx2e} % provides \textsubscript
\ifnum 0\ifxetex 1\fi\ifluatex 1\fi=0 % if pdftex
  \usepackage[T1]{fontenc}
  \usepackage[utf8]{inputenc}
  \usepackage{textcomp} % provides euro and other symbols
\else % if luatex or xelatex
  \usepackage{unicode-math}
  \defaultfontfeatures{Ligatures=TeX,Scale=MatchLowercase}
\fi
% use upquote if available, for straight quotes in verbatim environments
\IfFileExists{upquote.sty}{\usepackage{upquote}}{}
% use microtype if available
\IfFileExists{microtype.sty}{%
\usepackage[]{microtype}
\UseMicrotypeSet[protrusion]{basicmath} % disable protrusion for tt fonts
}{}
\IfFileExists{parskip.sty}{%
\usepackage{parskip}
}{% else
\setlength{\parindent}{0pt}
\setlength{\parskip}{6pt plus 2pt minus 1pt}
}
\usepackage{hyperref}
\hypersetup{
            pdftitle={A2- Exxercise1 and 2},
            pdfauthor={geoffrey},
            pdfborder={0 0 0},
            breaklinks=true}
\urlstyle{same}  % don't use monospace font for urls
\usepackage[margin=1in]{geometry}
\usepackage{color}
\usepackage{fancyvrb}
\newcommand{\VerbBar}{|}
\newcommand{\VERB}{\Verb[commandchars=\\\{\}]}
\DefineVerbatimEnvironment{Highlighting}{Verbatim}{commandchars=\\\{\}}
% Add ',fontsize=\small' for more characters per line
\usepackage{framed}
\definecolor{shadecolor}{RGB}{248,248,248}
\newenvironment{Shaded}{\begin{snugshade}}{\end{snugshade}}
\newcommand{\AlertTok}[1]{\textcolor[rgb]{0.94,0.16,0.16}{#1}}
\newcommand{\AnnotationTok}[1]{\textcolor[rgb]{0.56,0.35,0.01}{\textbf{\textit{#1}}}}
\newcommand{\AttributeTok}[1]{\textcolor[rgb]{0.77,0.63,0.00}{#1}}
\newcommand{\BaseNTok}[1]{\textcolor[rgb]{0.00,0.00,0.81}{#1}}
\newcommand{\BuiltInTok}[1]{#1}
\newcommand{\CharTok}[1]{\textcolor[rgb]{0.31,0.60,0.02}{#1}}
\newcommand{\CommentTok}[1]{\textcolor[rgb]{0.56,0.35,0.01}{\textit{#1}}}
\newcommand{\CommentVarTok}[1]{\textcolor[rgb]{0.56,0.35,0.01}{\textbf{\textit{#1}}}}
\newcommand{\ConstantTok}[1]{\textcolor[rgb]{0.00,0.00,0.00}{#1}}
\newcommand{\ControlFlowTok}[1]{\textcolor[rgb]{0.13,0.29,0.53}{\textbf{#1}}}
\newcommand{\DataTypeTok}[1]{\textcolor[rgb]{0.13,0.29,0.53}{#1}}
\newcommand{\DecValTok}[1]{\textcolor[rgb]{0.00,0.00,0.81}{#1}}
\newcommand{\DocumentationTok}[1]{\textcolor[rgb]{0.56,0.35,0.01}{\textbf{\textit{#1}}}}
\newcommand{\ErrorTok}[1]{\textcolor[rgb]{0.64,0.00,0.00}{\textbf{#1}}}
\newcommand{\ExtensionTok}[1]{#1}
\newcommand{\FloatTok}[1]{\textcolor[rgb]{0.00,0.00,0.81}{#1}}
\newcommand{\FunctionTok}[1]{\textcolor[rgb]{0.00,0.00,0.00}{#1}}
\newcommand{\ImportTok}[1]{#1}
\newcommand{\InformationTok}[1]{\textcolor[rgb]{0.56,0.35,0.01}{\textbf{\textit{#1}}}}
\newcommand{\KeywordTok}[1]{\textcolor[rgb]{0.13,0.29,0.53}{\textbf{#1}}}
\newcommand{\NormalTok}[1]{#1}
\newcommand{\OperatorTok}[1]{\textcolor[rgb]{0.81,0.36,0.00}{\textbf{#1}}}
\newcommand{\OtherTok}[1]{\textcolor[rgb]{0.56,0.35,0.01}{#1}}
\newcommand{\PreprocessorTok}[1]{\textcolor[rgb]{0.56,0.35,0.01}{\textit{#1}}}
\newcommand{\RegionMarkerTok}[1]{#1}
\newcommand{\SpecialCharTok}[1]{\textcolor[rgb]{0.00,0.00,0.00}{#1}}
\newcommand{\SpecialStringTok}[1]{\textcolor[rgb]{0.31,0.60,0.02}{#1}}
\newcommand{\StringTok}[1]{\textcolor[rgb]{0.31,0.60,0.02}{#1}}
\newcommand{\VariableTok}[1]{\textcolor[rgb]{0.00,0.00,0.00}{#1}}
\newcommand{\VerbatimStringTok}[1]{\textcolor[rgb]{0.31,0.60,0.02}{#1}}
\newcommand{\WarningTok}[1]{\textcolor[rgb]{0.56,0.35,0.01}{\textbf{\textit{#1}}}}
\usepackage{graphicx,grffile}
\makeatletter
\def\maxwidth{\ifdim\Gin@nat@width>\linewidth\linewidth\else\Gin@nat@width\fi}
\def\maxheight{\ifdim\Gin@nat@height>\textheight\textheight\else\Gin@nat@height\fi}
\makeatother
% Scale images if necessary, so that they will not overflow the page
% margins by default, and it is still possible to overwrite the defaults
% using explicit options in \includegraphics[width, height, ...]{}
\setkeys{Gin}{width=\maxwidth,height=\maxheight,keepaspectratio}
\setlength{\emergencystretch}{3em}  % prevent overfull lines
\providecommand{\tightlist}{%
  \setlength{\itemsep}{0pt}\setlength{\parskip}{0pt}}
\setcounter{secnumdepth}{0}
% Redefines (sub)paragraphs to behave more like sections
\ifx\paragraph\undefined\else
\let\oldparagraph\paragraph
\renewcommand{\paragraph}[1]{\oldparagraph{#1}\mbox{}}
\fi
\ifx\subparagraph\undefined\else
\let\oldsubparagraph\subparagraph
\renewcommand{\subparagraph}[1]{\oldsubparagraph{#1}\mbox{}}
\fi

% set default figure placement to htbp
\makeatletter
\def\fps@figure{htbp}
\makeatother


\title{A2- Exxercise1 and 2}
\author{geoffrey}
\date{February 29, 2020}

\begin{document}
\maketitle

\hypertarget{exercise-1}{%
\subsection{Exercise 1}\label{exercise-1}}

\hypertarget{a.-randomized-design}{%
\subsubsection{a. Randomized design}\label{a.-randomized-design}}

A randomized design with two categorical factors, with

\begin{enumerate}
\def\labelenumi{\arabic{enumi}.}
\tightlist
\item
  the first factor having three categorical levels and
\item
  the second factor having two levels and
\item
  having three samples for each unique categorie
\end{enumerate}

can be produced with the following R code:

\begin{Shaded}
\begin{Highlighting}[]
\NormalTok{I=}\DecValTok{3}\NormalTok{; J=}\DecValTok{2}\NormalTok{; N=}\DecValTok{3}
\KeywordTok{rbind}\NormalTok{(}\KeywordTok{rep}\NormalTok{(}\DecValTok{1}\OperatorTok{:}\NormalTok{I,}\DataTypeTok{each=}\NormalTok{N}\OperatorTok{*}\NormalTok{J),}\KeywordTok{rep}\NormalTok{(}\DecValTok{1}\OperatorTok{:}\NormalTok{J,N}\OperatorTok{*}\NormalTok{I),}\KeywordTok{sample}\NormalTok{(}\DecValTok{1}\OperatorTok{:}\NormalTok{(N}\OperatorTok{*}\NormalTok{I}\OperatorTok{*}\NormalTok{J)))}
\end{Highlighting}
\end{Shaded}

\hypertarget{b.-plotting}{%
\subsubsection{b. Plotting}\label{b.-plotting}}

The boxplot and interaction plot below confirms our intuition:

\begin{enumerate}
\def\labelenumi{\arabic{enumi}.}
\tightlist
\item
  A cold environment causes a much slower decay
\item
  Wet bread has a much wider distribution (variance)
\item
  On average dry bread decays slower than wet bread
\item
  However, wet and cold (frozen) bread has the slowest decay
\end{enumerate}

From the non-parallel lines in the interaction plot and the wide
distribution of the wet sample we conclude that the (wet) humidity
amplifies the effect of the temperature and thus it can be explained by
the strong interaction between the two factors (opposed to the errors in
the measurement).

\includegraphics{A2-exercise1-2_files/figure-latex/unnamed-chunk-2-1.pdf}

\includegraphics{A2-exercise1-2_files/figure-latex/unnamed-chunk-3-1.pdf}

\hypertarget{c.-two-way-anova}{%
\subsubsection{c. Two way ANOVA}\label{c.-two-way-anova}}

We have 3 hyphotheses here: H0:There is no main effect of first factor
(humidity) H0:There is no main effect of second factor (environment)
H0:There is no interactions between two factors From the two-way anova
resulst below, we reject both hyphotheses which means both factors have
a main effect on the decay time of bread, and the factors have an
interaction effect.

\begin{verbatim}
## Analysis of Variance Table
## 
## Response: hours
##                      Df Sum Sq Mean Sq F value    Pr(>F)    
## humidity              1  26912   26912  62.296 4.316e-06 ***
## environment           2 201904  100952 233.685 2.461e-10 ***
## humidity:environment  2  55984   27992  64.796 3.705e-07 ***
## Residuals            12   5184     432                      
## ---
## Signif. codes:  0 '***' 0.001 '**' 0.01 '*' 0.05 '.' 0.1 ' ' 1
\end{verbatim}

\hypertarget{d.-coefficients-need-review}{%
\subsubsection{d. Coefficients -- need review
---}\label{d.-coefficients-need-review}}

According to the means of squares, on avergae the environment has the
biggest effect on the decay. However this can not be concluded so
easily, because it is being compared to one base (the first categorie),
instead of a more comprehensive annalysis.

\begin{verbatim}
##                                     Estimate Std. Error    t value     Pr(>|t|)
## (Intercept)                              364   12.00000  30.333333 1.032769e-12
## humiditywet                               72   16.97056   4.242641 1.142103e-03
## environmentintermediate                 -124   16.97056  -7.306770 9.389760e-06
## environmentwarm                         -100   16.97056  -5.892557 7.336887e-05
## humiditywet:environmentintermediate     -180   24.00000  -7.500000 7.233671e-06
## humiditywet:environmentwarm             -268   24.00000 -11.166667 1.073751e-07
\end{verbatim}

\hypertarget{e.-diagnostics}{%
\subsubsection{e. Diagnostics}\label{e.-diagnostics}}

The first requirements is that for each unique categorie, there should
be at least 2 samples, which is the case. Then the most important
requirement is that the data among the factors should approximiatly have
equal variances. This has been tested in b). and the conclusions was
that they approximitally were the same. A different test we can do after
the ANOVA test, is check whether the error is normally distributed,
which is to be expected of a random variable. In the following QQplot it
can be seen that the residuals are approximiatly normally distributed.
And in the fitted residuals plot it can be seen that the spread is
approxomitelly horizentally symmetric among the fitted values, however
there are 2 outliers in the middle.

\includegraphics{A2-exercise1-2_files/figure-latex/unnamed-chunk-6-1.pdf}

\hypertarget{exercise-2}{%
\subsection{Exercise 2}\label{exercise-2}}

\hypertarget{a.-randomized-block-design}{%
\subsubsection{a. Randomized block
design}\label{a.-randomized-block-design}}

The following code generates a random block design with five blocks, a
factor with three levels, and one sample per unique categorie.

\begin{Shaded}
\begin{Highlighting}[]
\NormalTok{B=}\DecValTok{5}\NormalTok{;}
\NormalTok{if1 =}\StringTok{ }\KeywordTok{sample}\NormalTok{(}\DecValTok{1}\OperatorTok{:}\DecValTok{5}\NormalTok{)}
\NormalTok{if2 =}\StringTok{ }\KeywordTok{sample}\NormalTok{(}\DecValTok{6}\OperatorTok{:}\DecValTok{10}\NormalTok{)}
\NormalTok{if3 =}\StringTok{ }\KeywordTok{sample}\NormalTok{(}\DecValTok{11}\OperatorTok{:}\DecValTok{15}\NormalTok{)}
\ControlFlowTok{for}\NormalTok{ (i }\ControlFlowTok{in} \DecValTok{1}\OperatorTok{:}\NormalTok{B) }\KeywordTok{print}\NormalTok{(}\KeywordTok{c}\NormalTok{(if1[i], if2[i], if3[i]))}
\end{Highlighting}
\end{Shaded}

\hypertarget{b.-graphical-pre-analysis}{%
\subsubsection{b. Graphical
pre-analysis}\label{b.-graphical-pre-analysis}}

The boxplots below suggest that indeed the skil level and the interfaces
matter for the search time. Where skill level 1 is indeed the fastest,
and interface 1 is the fastest from the three interfaces. And from the
interaction plots below it can be observed that overall the factors have
the same pattern; all lines start in the lower left corner and end
towwards the upper right corner. However, they are not perfectly
parallel, this can be explained by the small sample sizes which causes
local irregularities. Thus we conclude that there is no interaction
between the two factors.

\includegraphics{A2-exercise1-2_files/figure-latex/unnamed-chunk-8-1.pdf}
\includegraphics{A2-exercise1-2_files/figure-latex/unnamed-chunk-8-2.pdf}

\hypertarget{c.-anova-with-1-block-factor}{%
\subsubsection{c. Anova with 1 block
factor}\label{c.-anova-with-1-block-factor}}

From the ANOVA results below it can be concluded that the search time is
not the same for all interfaces. Futhermore, we can estimate the time it
takes for a user with skill level 3 to find a product using interface 2
by looking at the summary table and adding the coeffcients of these two
categories to the incercept. Thus that would be 15.015+3.033+2.7=20.748.

\begin{verbatim}
## Analysis of Variance Table
## 
## Response: time
##           Df Sum Sq Mean Sq F value  Pr(>F)  
## interface  2 50.465 25.2327  7.8237 0.01310 *
## skill      4 80.051 20.0127  6.2052 0.01421 *
## Residuals  8 25.801  3.2252                  
## ---
## Signif. codes:  0 '***' 0.001 '**' 0.01 '*' 0.05 '.' 0.1 ' ' 1
\end{verbatim}

\begin{verbatim}
## 
## Call:
## lm(formula = time ~ interface + skill, data = search)
## 
## Residuals:
##     Min      1Q  Median      3Q     Max 
## -2.5733 -0.6967  0.3867  1.0567  1.7867 
## 
## Coefficients:
##             Estimate Std. Error t value Pr(>|t|)    
## (Intercept)   15.013      1.227  12.238 1.85e-06 ***
## interface2     2.700      1.136   2.377  0.04474 *  
## interface3     4.460      1.136   3.927  0.00438 ** 
## skill2         1.300      1.466   0.887  0.40118    
## skill3         3.033      1.466   2.069  0.07238 .  
## skill4         5.300      1.466   3.614  0.00684 ** 
## skill5         6.100      1.466   4.160  0.00316 ** 
## ---
## Signif. codes:  0 '***' 0.001 '**' 0.01 '*' 0.05 '.' 0.1 ' ' 1
## 
## Residual standard error: 1.796 on 8 degrees of freedom
## Multiple R-squared:  0.8349, Adjusted R-squared:  0.7111 
## F-statistic: 6.745 on 6 and 8 DF,  p-value: 0.008395
\end{verbatim}

\hypertarget{d.-diagnostics}{%
\subsubsection{d. Diagnostics}\label{d.-diagnostics}}

The QQ-plot of the residuals below looks normally distributed, which is
good. The fitted residuals do not depict any outliers.

\includegraphics{A2-exercise1-2_files/figure-latex/unnamed-chunk-10-1.pdf}

\hypertarget{e.-friedmann-test}{%
\subsubsection{e. Friedmann test}\label{e.-friedmann-test}}

The result of the Friedman test is the same as the ANOVA: we reject the
H0, thus there is a difference in sesarch times.

\begin{verbatim}
## 
##  Friedman rank sum test
## 
## data:  search$time, search$interface and search$skill
## Friedman chi-squared = 6.4, df = 2, p-value = 0.04076
\end{verbatim}

\hypertarget{f.-one-way-anova}{%
\subsubsection{f. One-way ANOVA}\label{f.-one-way-anova}}

The one-way ANOVA returns no significant difference in the search time
between the interfaces. This result is not very usefull, because 1) we
removed a lot of information from the model and 2) the model now assumes
that the block is a random selection of all available blocks, which is
not the case because the blocks were fixed/predetermined.

\begin{verbatim}
## Analysis of Variance Table
## 
## Response: time
##           Df  Sum Sq Mean Sq F value  Pr(>F)  
## interface  2  50.465  25.233  2.8605 0.09642 .
## Residuals 12 105.852   8.821                  
## ---
## Signif. codes:  0 '***' 0.001 '**' 0.01 '*' 0.05 '.' 0.1 ' ' 1
\end{verbatim}

\end{document}
